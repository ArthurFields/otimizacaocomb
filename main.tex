\documentclass[10pt,reqno]{amsart}

% README FIRST
%
% Todos os alunos podem contribuir com a elaboração e manutenção
% destas notas de aula, o que não só é de interesse para a
% solidificação do aprendizado, como deverá contribuir *muito
% positivamente* para seu fator alpha de participação.
%
% É de *extrema importância* manter o source code em bom estado.  A
% regra principal é *manter a consistência* da formatação e dos
% comandos usados.  Antes de discutirmos regras mais específicas
% abaixo, note que a violação dessas regras pode acarretar numa
% diminuição do seu alpha.  Isso não significa que, se você fizer
% algum push que viole as regras, você será penalizado.  Mas se você
% for descuidado com o arquivo, e não consertar os erros com uma certa
% prontidão mesmo após ser notificado, você poderá ser penalizado por
% atrapalhar os colegas.
%
% 1. Não use $ para entrar e $ para sair de inline math mode.  Use \(
% e \); veja https://tex.stackexchange.com/questions/510/are-and-preferable-to-dollar-signs-for-math-mode
%
% 2. Não invente comandos de notação que já existem no arquivo
% notation.sty.  Use os comandos lá definidos.
%
% 3. Não adicione soluções para os exercícios, pois eles podem ser
% pedidos em listas.
%
% TODO:
%
% a. As notas estão num estado meio ridículo, com inglês e português
% misturado.  Uma ótima forma de contribuir seria completar a tradução
% do que já foi escrito em inglês.

\usepackage{packages}

\title{MAC0325 Combinatorial Optimization: Lecture Notes}
\author{Instrutor: Marcel {K.} de Carli Silva}
\date{2º semestre de 2018}
\renewcommand{\datename}{\emph{Data:}}
\address{%
  Instituto de Matemática e Estatística, %
  Universidade de São Paulo, %
  R. do Matão 1010, 05508-090, São Paulo, SP}
%
\email{mksilva@ime.usp.br}

\begin{document}

\begin{abstract}

\end{abstract}

\maketitle

\tableofcontents

\section{Aula de terça-feira, 7 de agosto}

\section{Aula de quinta-feira, 9 de agosto}

\section{Aula de terça-feira, 14 de agosto}

To solve an optimization problem means to determine whether it is
\begin{enumerate}[(i)]
\item infeasible, or
\item unbounded, or
\item determine the optimal value and find an optimal solution, if one
  exists.
\end{enumerate}

\begin{problem*}[Problema do Caminho Mínimo]
  Dado um digrafo \(D = (V,A,\varphi)\), vértices \(r,s \in V\) e
  \(\ell \colon A \to \Reals\), resolva o problema de otimização
  \begin{equation}
    \label{eq:1}
    \tag{SPP}
    \begin{array}{{rl}}
      \text{Minimizar} & \ell(P) \\
      \text{sujeito a} & \text{\(P\) é um caminho de \(r\) a \(s\) em~\(D\)}.
    \end{array}
  \end{equation}
\end{problem*}

\begin{problem*}[The Shortest Walks Problem]
  Given a digraph \(D = (V,A,\varphi)\), vertices \(r,s \in V\), and
  \(\ell \colon A \to \Reals\), solve the optimization problem
  \begin{equation}
    \label{eq:2}
    \tag{SWP}
    \begin{array}{{rl}}
      \text{Minimize}   & \ell(P) \\
      \text{subject to} & \text{\(P\) is a walk from \(r\) to \(s\) in~\(D\)}.
    \end{array}
  \end{equation}
\end{problem*}

\begin{definition*}
  Let \((X,f)\) and \((Y,g)\) be optimization problems.  A
  \emph{homomorphism} from \((X,f)\) to \((Y,g)\) is a function
  \(\psi \colon X \to Y\) such that \(g(\psi(x)) \leq f(x)\) for each
  \(x \in X\).  If there exists a homomorphism from \((X,f)\) to
  \((Y,g)\), we write \((X,f) \to (Y,g)\).  The problems \((X,f)\) and
  \((Y,g)\) are \emph{equivalent} if \((X,f) \to (Y,g)\) and
  \((Y,g) \to (X,f)\).
\end{definition*}

\begin{exercise}
  \label{ex:1}
  Prove that either problem \eqref{eq:1} or \eqref{eq:2} is equivalent
  to the same problem applied to the induced subdigraph \(D[V_{rs}]\)
  where
  \(V_{rs} \coloneqq \setst{v \in V}{r \reaches_D v \text{ and }v
    \reaches_D s}\).
\end{exercise}

\begin{exercise}
  Prove that, if two optimization problems are equivalent, they both
  have one of the possible outcomes: (i) infeasible; (ii) unbounded;
  (iii) finite optimal value but not optimal solution; (iv) finite
  optimal value and there exists an optimal solution.
\end{exercise}

\Cref{ex:1} shows that very little generality is lost in either
\eqref{eq:1} or \eqref{eq:2} if we assume that,
\begin{equation}
  \label{eq:3}
  \text{for each \(v \in V\), we have \(r \reaches_D v \reaches_D s\).}
\end{equation}
It is also obvious how to reduce any instance of either problem to one
where the digraph is simple.

\begin{exercise}
  \label{ex:2}
  Let \(D = (V,A)\) be a digraph, let \(r,s \in V\), and let
  \(\ell \colon A \to \Reals\).  Prove that, if some optimal solution
  \(W^\star\) of~\eqref{eq:2} is a path, then \(W^\star\) is an
  optimal solution of~\eqref{eq:1}.
\end{exercise}

\begin{exercise}
  \label{ex:3}
  Let \(D = (V,A)\) be a digraph, let \(r,s \in V\), and let
  \(\ell \colon A \to \Reals\).  If
  \(\seq{v_0,a_1,v_1,\dotsc,a_k,v_k}\) is an optimal solution
  for~\eqref{eq:2} and \(v_i = v_j\) for some
  \(i,j \in \set{0,\dotsc,k}\) such that \(i < j\), then
  \(C \coloneqq \seq{v_i,a_{i+1},v_{i+1},\dotsc,a_j,v_j}\) costs
  \(\ell(C) \geq 0\).
\end{exercise}

\begin{proposition}
  \label{prop:1}
  Let \(D = (V,A)\) be a digraph, let \(r,s \in V\), and let
  \(\ell \colon A \to \Reals\).  Then any optimal solution
  to~\eqref{eq:2} of minimum length is a path, and hence an optimal
  solution to~\eqref{eq:1}.
\end{proposition}
\begin{proof}
  Let \(W = \seq{v_0,a_1,v_1,\dotsc,a_k,v_k}\) be an optimal solution
  for~\eqref{eq:2} with the minimum number of arcs.  If there exist
  \(i,j \in \set{0,\dotsc,k}\) such that \(v_i = v_j\) and \(i < j\),
  then
  \(W' \coloneqq
  \seq{v_0,a_1,v_1,\dotsc,a_i,v_i,a_{j+1},v_{j+1},\dotsc,a_k,v_k}\) is
  a walk from~\(r\) to~\(s\) with \(\ell(W') = \ell(W) - \ell(C)\) for
  \(C \coloneqq \seq{v_i,a_{i+1},v_{i+1},\dotsc,a_j,v_j}\) so
  \(\ell(W') \leq \ell(W)\) by \Cref{ex:3}.  By optimality,
  \(\ell(W') = \ell(W)\) and \(W'\) has smaller length than~\(W\), a
  contradiction.  Hence, \(W\) is a path.  We are done by \Cref{ex:2}.
\end{proof}

Let \(D = (V,A,\varphi)\) be a digraph.  Define
\begin{alignat*}{2}
  \label{eq:4}
  \deltaout[D](S)
  & \coloneqq
  \setst{
    a \in A
  }{
    \varphi(a) = uv,\,
    u \in S,\,
    v \not\in S
  }
  & \qquad &
  \forall S \subseteq V,
  \\
  \deltain[D](S)
  & \coloneqq
  \deltaout[D](V \setminus S)
  & \qquad &
  \forall S \subseteq V.
\end{alignat*}
Define the binary relation \(\reaches_D\) on~\(V\) by
\begin{equation*}
  r \reaches_D s
  \text{ if there is a walk from~\(r\) to~\(s\) in~\(D\)}.
\end{equation*}

\begin{theorem}
  Let \(D = (V,A)\) be a digraph and let \(r,s \in V\).  Then
  \(r \not\reaches_D s \iff\) there exists \(R \subseteq V\) such that
  \(r \in R\), \(s \not\in R\) and \(\deltaout[D](R) = \emptyset\).
\end{theorem}
\begin{proof}
  
\end{proof}

\section{Aula de quinta-feira, 16 de agosto}

\begin{proposition}
  \label{prop:2}
  Let \(D = (V,A)\) be a digraph, let \(r,s \in V\), and let
  \(\ell \colon A \to \Reals\).  Suppose that~\eqref{eq:3} holds.
  Then
  \begin{equation}
    \label{eq:5}
    \text{\eqref{eq:2} is unbounded}
    \iff
    \text{\(D\) has a negative circuit}.
  \end{equation}
\end{proposition}
\begin{proof}[Proof Sketch]
  
\end{proof}

\begin{exercise}
  \label{ex:4}
  Let \(D = (V,A)\) be a digraph, let \(r,s \in V\), and let
  \(\ell \colon A \to \Reals\).  Suppose that~\eqref{eq:3} holds.
  Prove that, if the optimal value of~\eqref{eq:2} is finite, then
  there exists an optimal solution.
\end{exercise}

Sejam \(D = (V,A)\) um digrafo, \(r \in V\) e
\(\ell \colon A \to \Reals\).  Para cada \(t \in \Naturals\), defina
\(y_t \colon V \to \Reals \cup \set{\infty}\) como
\begin{equation}
  \label{eq:6}
  y_0(v)
  \coloneqq
  \begin{cases*}
    0       & se \(v = r\),   \\
    +\infty & caso contrário, \\
  \end{cases*}
\end{equation}
e
\begin{equation}
  \label{eq:7}
  y_{t+1}(v)
  \coloneqq
  \min\paren[\big]{
    \set{y_t(v)}
    \cup
    \setst{
      y_t(u) + \ell(uv)
    }{
      uv \in A
    }}
  \qquad
  \forall t \in \Naturals.
\end{equation}
Para cada \(t \in \Naturals\) e \(v \in V\), sejam
\begin{equation}
  \label{eq:8}
  W_0(v)
  \coloneqq
  \begin{cases*}
    \seq{r} & se \(v = r\),   \\
    \bot    & caso contrário, \\
  \end{cases*}
\end{equation}
e
\begin{equation}
  \label{eq:9}
  W_{t+1}(v)
  \coloneqq
  \begin{cases*}
    \seq{z_0,\dotsc,a_k,z_k,uv,v} & se \(y_t(u) + \ell(uv) =
    y_{t+1}(v) < y_t(v)\) e \(W_t(u) = \seq{z_0,\dotsc,a_k,z_k}\), \\
    W_t(v) & se \(y_{t+1}(v) = y_t(v)\) \\
  \end{cases*}
  \quad
  \forall t \in \Naturals.
\end{equation}
Note que \(y_t\) está definido inambiguamente para todo
\(t \in \Naturals\), enquanto os passeios \(W_t(v)\) não são definidos
de forma única.  Uma possibilidade para que esses passeios estejam
unicamente definidos é criar uma ordem total arbitrária~\(\leq\)
sobre~\(V\) e, em caso de vários arcos \(uv\) tais que
\(y_{t+1}(v) = y_t(u) + \ell(uv)\), podemos sempre escolher o
elemento~\(u\) mínimo na ordem~\(\leq\).

\begin{exercise}
  \label{ex:5}
  Sejam \(D = (V,A)\) um digrafo, \(r \in V\) e
  \(\ell \colon A \to \Reals\).  Prove que, para cada
  \(t \in \Naturals\) e cada \(s \in V\), o~valor ótimo do problema de
  otimização
  \begin{equation}
    \tag{SWP\({}_t\)}
    \begin{array}{{rl}}
      \text{Minimize}   & \ell(P) \\
      \text{subject to} & \text{\(P\) é um passeio de \(r\) a \(s\) em~\(D\) de comprimento \(\leq t\)} \\
    \end{array}
  \end{equation}
  é \(y_t(s)\) e \(W_t(s)\) é uma solução ótima se
  \(y_t(s) < \infty\).
\end{exercise}

It is easy to check that \(y_t\) can be computed in time polynomial in
\(n = \card{V}\) and~\(t\).  A smart way to do this computationally
uses dynamic programming, and it is not hard to recover the optimal
walks from the recurrence relations in~\eqref{eq:6} and~\eqref{eq:7}.
Suppose that~\eqref{eq:3} holds.  By~\Cref{prop:1,prop:2}
and~\Cref{ex:4}, if \(D\) has no negative circuits, then
\(y_{n-1} = \dist_{(G,\ell)}(r,\cdot)\).  How can we detect/certify
whether \(D\) has negative circuits and certify that the walks that we
find are optimal?

Let \(D = (V,A)\) be a digraph and let \(\ell \colon A \to \Reals\).
Let us temporarily call a function
\(y \colon V \to \Reals \cup \set{\infty}\) a \emph{potential}.  A
potential is \emph{feasible} if
\begin{equation*}
  y(v) \leq y(u) + \ell(uv)
  \qquad
  \forall uv \in A.
\end{equation*}
If \(y\) is a feasible potential and \(y(r) < \infty\), then for every
\(v \in V\), we have \(r \reaches_D v \implies y(v) < \infty\).  So
there is very little loss of generality when assume that a feasible
potential is finite everywhere.

\begin{theorem}
  \label{thm:1}
  Let \(D = (V,A)\) be a digraph, let \(r,s \in V\), and let
  \(\ell \colon A \to \Reals\).  If \(y \colon V \to \Reals\) is a
  feasible potential, then
  \begin{equation}
    \label{eq:10}
    \ell(W) \geq y(s) - y(r)
    \qquad
    \text{for any walk~\(W\) from~\(r\) to~\(s\) in~\(D\)}.
  \end{equation}
\end{theorem}
\begin{proof}
  Let \(W = \seq{v_0,a_1,v_1,\dotsc,a_k,v_k}\) be a walk from
  \(r = v_0\) to \(s = v_k\) in~\(D\).  Then
  \begin{equation*}
    \ell(W)
    =
    \sum_{i=1}^k \ell(a_i)
    =
    \sum_{i=1}^k \ell(v_{i-1}v_i)
    \geq
    \sum_{i=1}^k \sqbrac[\big]{y(v_i)-y(v_{i-1})}
    =
    y(v_k) - y(v_0)
    =
    y(s) - y(r)
  \end{equation*}
  since the sum telescopes.
\end{proof}

\begin{corollary}
  Let \(D = (V,A)\) be a digraph and let \(\ell \colon A \to \Reals\).
  If there is a feasible potential \(y \colon V \to \Reals\) then
  \(D\) has no negative circuits.
\end{corollary}
\begin{proof}
  Let \(C\) be a circuit in~\(D\) from \(r\) to \(s = r\).  Then
  \(\ell(C) \geq y(s) - y(r) = 0\) by \Cref{thm:1}.
\end{proof}

\begin{corollary}
  \label{cor:1}
  Let \(D = (V,A)\) be a digraph on \(n\) vertices, let \(r \in V\),
  and let \(\ell \colon A \to \Reals\).  Suppose that, for each
  \(v \in V\), we have \(r \reaches_D v\).  If \(D\) has no negative
  circuit, then \(y_{n-1} = \dist_{(G,\ell)}(r,\cdot)\) is a feasible
  potential.
\end{corollary}
\begin{proof}
  This is a summary of what was discussed above.  Write the full proof
  as an exercise.
\end{proof}

Regarding the functions \(y_t\) defined in~\eqref{eq:6}
and~\eqref{eq:7}, it is easy to verify that \(y_{t} = y_{t+1}\) if and
only if \(y_t\) is a feasible potential.  Hence, to detect whether
\(D\) has a negative circuit, it suffices to check whether
\(y_{n-1} = y_n\) by \Cref{cor:1}.  To actually find a negative
circuit, suppose that \(y_n \neq y_{n-1}\).  Since
\(y_n \leq y_{n-1}\), there exists \(v \in V\) such that
\(y_n(v) < y_{n-1}(v)\).  By~\Cref{ex:5}, there is a walk~\(W\)
from~\(r\) to~\(v\) in~\(D\) of length exactly~\(n\) and cost
\(\ell(W) < \dist_{(G,\ell)}(r,v)\).  Hence, \(W\) must ``contain'' a
negative circuit.

\begin{theorem}[O algoritmo de Bellman-Ford]
  Sejam \(D = (V,A)\) um digrafo, \(r,s \in V\) e
  \(\ell \colon A \to \Reals\).  Defina \(y_t\) para cada
  \(t \in \Naturals\) como em~\eqref{eq:6} e~\eqref{eq:7} e defina
  \(W_t(v)\) para cada \(t \in \Naturals\) e \(v \in V\) como
  em~\eqref{eq:8} e~\eqref{eq:9}.  Defina
  \(R \coloneqq \setst{v \in V}{r \reaches_D v}\) e
  \(S \coloneqq \setst{v \in V}{v \reaches_D s}\).  Defina
  \(\bar{y} \coloneqq y_{n-1}\restrict{R \cap S}\), onde
  \(n \coloneqq \card{V}\).  Então vale precisamente uma das
  afirmações abaixo:
  \begin{enumerate}[(i)]
  \item O problema~\eqref{eq:2} é inviável e
    \(R = \setst{v \in V}{\bar{y}_v < \infty}\) satisfaz \(r \in R\),
    \(s \not\in R\) e \(\deltaout(R) = \emptyset\).
  \item O problema~\eqref{eq:2} é ilimitado e existe \(v \in V\) tal
    que o passeio \(W_n(v) = \seq{v_0,a_1,v_1,\dotsc,a_n,v_n}\) tem
    comprimento~\(n\) e, para quaisquer \(i,j \in \set{0,\dotsc,n}\)
    com \(i < j\) e \(v_i = v_j\), o passeio
    \(W_{ij} \coloneqq \seq{v_i,\dotsc,v_j}\) satisfaz
    \(\ell(W_{ij}) < 0\).
  \item O problema~\eqref{eq:2} tem valor ótimo
    \(\bar{y}(s) \in \Reals\), o caminho \(W_{n-1}(s)\) é solução
    ótima para~\eqref{eq:2} e~\eqref{eq:1}, e \(\bar{y}\) é um
    potencial viável.
  \end{enumerate}
\end{theorem}
\begin{proof}
  Novamente este é um resumo do que foi discutido acima.  Escreva a
  prova completa como exercício.
\end{proof}

\nocite{CookCPS98a,Schrijver03a}

\begingroup
\printbibliography
\endgroup

\end{document}

%%% Local Variables:
%%% mode: latex
%%% TeX-master: t
%%% coding: utf-8
%%% End:
