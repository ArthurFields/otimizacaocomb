\documentclass[10pt,reqno]{amsart}

% README FIRST
%
% Todos os alunos podem contribuir com a elaboração e manutenção
% destas notas de aula, o que não só é de interesse para a
% solidificação do aprendizado, como deverá contribuir *muito
% positivamente* para seu fator alpha de participação.
%
% É de *extrema importância* manter o source code em bom estado.  A
% regra principal é *manter a consistência* da formatação e dos
% comandos usados.  Antes de discutirmos regras mais específicas
% abaixo, note que a violação dessas regras pode acarretar numa
% diminuição do seu alpha.  Isso não significa que, se você fizer
% algum push que viole as regras, você será penalizado.  Mas se você
% for descuidado com o arquivo, e não consertar os erros com uma certa
% prontidão mesmo após ser notificado, você poderá ser penalizado por
% atrapalhar os colegas.
%
% 1. Não use $ para entrar e $ para sair de inline math mode.  Use \(
% e \); veja https://tex.stackexchange.com/questions/510/are-and-preferable-to-dollar-signs-for-math-mode
%
% 2. Não invente comandos de notação que já existem no arquivo
% notation.sty.  Use os comandos lá definidos.
%
% 3. Não adicione soluções para os exercícios, pois eles podem ser
% pedidos em listas.
%
% TODO:
%
% a. As notas estão num estado meio ridículo, com inglês e português
% misturado.  Uma ótima forma de contribuir seria completar a tradução
% do que já foi escrito em inglês.

\usepackage{packages}

\title{MAC0325 Combinatorial Optimization: Lecture Notes}
\author{Instrutor: Marcel {K.} de Carli Silva}
\date{2º semestre de 2018}
\renewcommand{\datename}{\emph{Data:}}
\address{%
  Instituto de Matemática e Estatística, %
  Universidade de São Paulo, %
  R. do Matão 1010, 05508-090, São Paulo, SP}
%
\email{mksilva@ime.usp.br}

\begin{document}

\begin{abstract}

\end{abstract}

\maketitle

\tableofcontents

\section{Aula de terça-feira, 7 de agosto}

Um \emph{problema de otimização} é um par ordenado \((X,f)\) onde
\(X\) é um conjunto, chamado \emph{conjunto viável} ou \emph{região
  viável}, e \(f \colon X \to \Reals\) é uma função, chamada
\emph{função objetivo}.  Mais comumente, nos referimos ao problema de
otimização \((X,f)\) como o problema:
\begin{equation}
  \label{eq:1}
  \tag{OPT}
  \begin{array}{{rl}}
    \text{Minimizar} & f(x)     \\
    \text{sujeito a} & x \in X. \\
  \end{array}
\end{equation}
Cada elemento de \(X\) é uma \emph{solução viável} de~\eqref{eq:1} e
seu \emph{valor objetivo} é \(f(x)\).  O \emph{valor ótimo}
de~\eqref{eq:1} é \(\inf_{x \in X}f(x) \in [-\infty,+\infty]\).  O
problema é \emph{inviável} se \(X = \emptyset\).  Note que o valor
ótimo de um problema é \(+\infty\) se, e somente se, o problema é
inviável.  Se o valor ótimo é \(-\infty\), o problema é
\emph{ilimitado}.  Uma \emph{solução ótima} é uma solução viável cujo
valor objetivo é o valor ótimo do problema; equivalentemente,
\(x^{\star} \in X\) é solução ótima se, e somente se, para todo
\(x \in X\), vale que \(f(x^{\star}) \leq f(x)\).

\section{Aula de quinta-feira, 9 de agosto}

\section{Aula de terça-feira, 14 de agosto}

\emph{Resolver um problema de otimização} significa:
\begin{enumerate}[(i)]
\item determinar se ele é inviável, ou
\item determinar se ele é ilimitado, ou
\item determinar o valor ótimo e achar uma solução ótima, se existir.
\end{enumerate}

\begin{problem*}[Problema do Caminho Mínimo]
  Dado um digrafo \(D = (V,A,\varphi)\), vértices \(r,s \in V\) e
  \(\ell \colon A \to \Reals\), resolva o problema de otimização:
  \begin{equation}
    \label{eq:2}
    \tag{SPP}
    \begin{array}{{rl}}
      \text{Minimizar} & \ell(P) \\
      \text{sujeito a} & \text{\(P\) é um caminho de \(r\) a \(s\) em~\(D\)}.
    \end{array}
  \end{equation}
\end{problem*}

\begin{problem*}[Problema do Passeio Mínimo]
  Dados um digrafo \(D = (V,A,\varphi)\), vértices \(r,s \in V\) e
  \(\ell \colon A \to \Reals\), resolva o problema de otimização:
  \begin{equation}
    \label{eq:3}
    \tag{SWP}
    \begin{array}{{rl}}
      \text{Minimizar}   & \ell(P) \\
      \text{sujeito a} & \text{\(P\) é um passeio de \(r\) a \(s\) em~\(D\)}.
    \end{array}
  \end{equation}
\end{problem*}

\begin{definition*}
  Sejam \((X,f)\) e \((Y,g)\) problemas de otimização.  Um
  \emph{homomorfismo} de \((X,f)\) a \((Y,g)\) é uma função
  \(\psi \colon X \to Y\) tal que \(g(\psi(x)) \leq f(x)\) para todo
  \(x \in X\).  Se existe um homomorfismo de \((X,f)\) a
  \((Y,g)\), escrevemos \((X,f) \to (Y,g)\).  Os problemas \((X,f)\) e
  \((Y,g)\) são \emph{equivalentes} se \((X,f) \to (Y,g)\) e
  \((Y,g) \to (X,f)\).
\end{definition*}

\begin{exercise}
  \label{ex:1}
  Prove que tanto o problema \eqref{eq:1} quanto \eqref{eq:2} são     
  ao mesmo problema aplicado ao subdigrafo induzido \(D[V_{rs}]\)
  em que
  \(V_{rs} \coloneqq \setst{v \in V}{r \reaches_D v \text{ e }v
    \reaches_D s}\).
\end{exercise}

\begin{exercise}
  Prove que, se dois problemas de otimização são equivalentes, então
  ambos têm um desses possíveis resultados: (i) inviável; (ii) ilimitado;
  (iii) valor ótimo finito mas sem solução ótima; (iv) valor ótimo finito
  e existe uma solução ótima.
\end{exercise}

\Cref{ex:1} mostra que pouca generalidade é perdida tanto em
\eqref{eq:2} quanto em \eqref{eq:3} se assumimos que,
\begin{equation}
  \label{eq:4}
  \text{para cada \(v \in V\), temos \(r \reaches_D v \reaches_D s\).}
\end{equation}
Também é óbvio como reduzir qualquer instância de ambos os problemas para 
uma em que o digrafo seja simples.

\begin{exercise}
  \label{ex:2}
  Sejam \(D = (V,A)\) um digrafo, \(r,s \in V\) e
  \(\ell \colon A \to \Reals\).  Prove que, se uma determinada solução
  ótima \(W^\star\) para~\eqref{eq:2} é um caminho, então \(W^\star\) é uma
  solução ótima para~\eqref{eq:1}.
\end{exercise}

\begin{exercise}
  \label{ex:3}
  Sejam \(D = (V,A)\) um digrafo, \(r,s \in V\) e
  \(\ell \colon A \to \Reals\).  Se
  \(\seq{v_0,a_1,v_1,\dotsc,a_k,v_k}\) é uma solução ótima
  para~\eqref{eq:2} e \(v_i = v_j\) para alguns
  \(i,j \in \set{0,\dotsc,k}\) tais que \(i < j\), então
  \(C \coloneqq \seq{v_i,a_{i+1},v_{i+1},\dotsc,a_j,v_j}\) tem custo
  \(\ell(C) \geq 0\).
\end{exercise}

\begin{proposition}
  \label{prop:1}
  Sejam \(D = (V,A)\) um digrafo, \(r,s \in V\) e
  \(\ell \colon A \to \Reals\).  Então qualquer solução ótima
  de~\eqref{eq:3} de comprimento mínimo é um caminho e, logo, uma
  solução ótima para~\eqref{eq:2}.
\end{proposition}
\begin{proof}
  Seja \(W = \seq{v_0,a_1,v_1,\dotsc,a_k,v_k}\) uma solução ótima
  para~\eqref{eq:2} com número de arcos mínimo.  Se existem
  \(i,j \in \set{0,\dotsc,k}\) tais que \(v_i = v_j\) e \(i < j\),
  então
  \(W' \coloneqq
  \seq{v_0,a_1,v_1,\dotsc,a_i,v_i,a_{j+1},v_{j+1},\dotsc,a_k,v_k}\) é
  um passeio de~\(r\) para~\(s\) com \(\ell(W') = \ell(W) - \ell(C)\) em que
  \(C \coloneqq \seq{v_i,a_{i+1},v_{i+1},\dotsc,a_j,v_j}\). Logo,
  \(\ell(W') \leq \ell(W)\) pelo \Cref{ex:3}.  Como \(W\) é solução ótima,
  \(\ell(W') = \ell(W)\) e \(W'\) tem comprimento menor do que~\(W\), o que
  é uma contradição.  Assim, \(W\) é um caminho.  O resultado agora sai
  diretamente do \Cref{ex:2}.
\end{proof}

Seja \(D = (V,A,\varphi)\) um digrafo.  Defina:
\begin{alignat*}{2}
  \label{eq:5}
  \deltaout[D](S)
  & \coloneqq
  \setst{
    a \in A
  }{
    \varphi(a) = uv,\,
    u \in S,\,
    v \not\in S
  }
  & \qquad &
  \forall S \subseteq V,
  \\
  \deltain[D](S)
  & \coloneqq
  \deltaout[D](V \setminus S)
  & \qquad &
  \forall S \subseteq V.
\end{alignat*}
Defina agora a relação binária \(\reaches_D\) em~\(V\) por
\begin{equation*}
  r \reaches_D s
  \text{ se existe um passeio de~\(r\) para~\(s\) em~\(D\)}.
\end{equation*}

\begin{theorem}
  Sejam \(D = (V,A)\) um digrafo e \(r,s \in V\).  Então
  \(r \not\reaches_D s \iff\) existe \(R \subseteq V\) tal que
  \(r \in R\), \(s \not\in R\) e \(\deltaout[D](R) = \emptyset\).
\end{theorem}
\begin{proof}
(\(\Leftarrow\)) Trivial.
\\ (\(\Rightarrow\)) Defina \(R \coloneqq \setst{v \in V}{r \reaches_D v}\).
   Note que \(r \in R\) e \(s \notin R\). Se \(\deltaout[D](R) \neq \emptyset\), 
   então existe \(v \in \deltaout[D](R)\). Assim, existe \(u \in R\) tal que
   \(\varphi(v) = uv\). Dessa forma, se \(W = \seq{r,a_1,v_1,\dotsc,a_k,u}\) 
   é um caminho de \(r\) a \(u\), então \(\seq{r,a_1,v_1,\dotsc,a_k,u,uv,v}\)
   é um caminho de \(r\) a \(v\). Contradição.
\end{proof}

\section{Aula de quinta-feira, 16 de agosto}

\begin{proposition}
  \label{prop:2}
  Sejam \(D = (V,A)\) um digrafo, \(r,s \in V\) e
  \(\ell \colon A \to \Reals\).  Suponha que vale~\eqref{eq:4}.
  Então
  \begin{equation}
    \label{eq:6}
    \text{\eqref{eq:3} é ilimitado}
    \iff
    \text{\(D\) tem um circuito negativo}.
  \end{equation}
\end{proposition}
\begin{proof} Desenho.
  
\end{proof}

\begin{exercise}
  \label{ex:4}
  Sejam \(D = (V,A)\) um digrafo, let \(r,s \in V\) e
  \(\ell \colon A \to \Reals\).  Suponha que vale~\eqref{eq:4}.
  Prove que, se o valor ótimo de~\eqref{eq:4} é finito, então
  existe uma solução ótima.
\end{exercise}

Sejam \(D = (V,A)\) um digrafo, \(r \in V\) e
\(\ell \colon A \to \Reals\).  Para cada \(t \in \Naturals\), defina
\(y_t \colon V \to \Reals \cup \set{\infty}\) como
\begin{equation}
  \label{eq:7}
  y_0(v)
  \coloneqq
  \begin{cases*}
    0       & se \(v = r\),   \\
    +\infty & caso contrário, \\
  \end{cases*}
\end{equation}
e
\begin{equation}
  \label{eq:8}
  y_{t+1}(v)
  \coloneqq
  \min\paren[\big]{
    \set{y_t(v)}
    \cup
    \setst{
      y_t(u) + \ell(uv)
    }{
      uv \in A
    }}
  \qquad
  \forall t \in \Naturals.
\end{equation}
Para cada \(t \in \Naturals\) e \(v \in V\), sejam
\begin{equation}
  \label{eq:9}
  W_0(v)
  \coloneqq
  \begin{cases*}
    \seq{r} & se \(v = r\),   \\
    \bot    & caso contrário, \\
  \end{cases*}
\end{equation}
e
\begin{equation}
  \label{eq:10}
  W_{t+1}(v)
  \coloneqq
  \begin{cases*}
    \seq{z_0,\dotsc,a_k,z_k,uv,v} & se \(y_t(u) + \ell(uv) =
    y_{t+1}(v) < y_t(v)\) e \(W_t(u) = \seq{z_0,\dotsc,a_k,z_k}\), \\
    W_t(v) & se \(y_{t+1}(v) = y_t(v)\) \\
  \end{cases*}
  \quad
  \forall t \in \Naturals.
\end{equation}
Note que \(y_t\) está definido inambiguamente para todo
\(t \in \Naturals\), enquanto os passeios \(W_t(v)\) não são definidos
de forma única.  Uma possibilidade para que esses passeios estejam
unicamente definidos é criar uma ordem total arbitrária~\(\leq\)
sobre~\(V\) e, em caso de vários arcos \(uv\) tais que
\(y_{t+1}(v) = y_t(u) + \ell(uv)\), podemos sempre escolher o
elemento~\(u\) mínimo na ordem~\(\leq\).

\begin{exercise}
  \label{ex:5}
  Sejam \(D = (V,A)\) um digrafo, \(r \in V\) e
  \(\ell \colon A \to \Reals\).  Prove que, para cada
  \(t \in \Naturals\) e cada \(s \in V\), o~valor ótimo do problema de
  otimização
  \begin{equation}
    \tag{SWP\({}_t\)}
    \begin{array}{{rl}}
      \text{Minimizar}   & \ell(P) \\
      \text{sujeito a} & \text{\(P\) é um passeio de \(r\) a \(s\) em~\(D\) de comprimento \(\leq t\)} \\
    \end{array}
  \end{equation}
  é \(y_t(s)\) e \(W_t(s)\) é uma solução ótima se
  \(y_t(s) < \infty\).
\end{exercise}

É fácil verificar que \(y_t\) pode ser computada em tempo polinomial em
\(n = \card{V}\) e~\(t\).  Um jeito esperto de fazer isso computacionalmente
usa programação dinâmica e não é difícil recuperar os passeios ótimos das 
relações de recorrência em~\eqref{eq:7} e~\eqref{eq:8}.
Suponha que vale~\eqref{eq:4}.  Pelas~\Cref{prop:1,prop:2}
e pelo~\Cref{ex:4}, se \(D\) não tem circuitos negativos, então
\(y_{n-1} = \dist_{(G,\ell)}(r,\cdot)\).  Como podemos detectar/certificar
se \(D\) tem circuitos negativos e certificar que os passeios que encontramos
são ótimos?

Sejam \(D = (V,A)\) um digrafo e \(\ell \colon A \to \Reals\).
Temporariamente, vamos chamar uma função
\(y \colon V \to \Reals \cup \set{\infty}\) de \emph{potencial}.  Um
potencial é \emph{viável} se
\begin{equation*}
  y(v) \leq y(u) + \ell(uv)
  \qquad
  \forall uv \in A.
\end{equation*}
If \(y\) is a feasible potential and \(y(r) < \infty\), then for every
\(v \in V\), we have \(r \reaches_D v \implies y(v) < \infty\).  So
there is very little loss of generality when assume that a feasible
potential is finite everywhere.

\begin{theorem}
  \label{thm:1}
  Let \(D = (V,A)\) be a digraph, let \(r,s \in V\), and let
  \(\ell \colon A \to \Reals\).  If \(y \colon V \to \Reals\) is a
  feasible potential, then
  \begin{equation}
    \label{eq:11}
    \ell(W) \geq y(s) - y(r)
    \qquad
    \text{for any walk~\(W\) from~\(r\) to~\(s\) in~\(D\)}.
  \end{equation}
\end{theorem}
\begin{proof}
  Let \(W = \seq{v_0,a_1,v_1,\dotsc,a_k,v_k}\) be a walk from
  \(r = v_0\) to \(s = v_k\) in~\(D\).  Then
  \begin{equation*}
    \ell(W)
    =
    \sum_{i=1}^k \ell(a_i)
    =
    \sum_{i=1}^k \ell(v_{i-1}v_i)
    \geq
    \sum_{i=1}^k \sqbrac[\big]{y(v_i)-y(v_{i-1})}
    =
    y(v_k) - y(v_0)
    =
    y(s) - y(r)
  \end{equation*}
  since the sum telescopes.
\end{proof}

\begin{corollary}
  Let \(D = (V,A)\) be a digraph and let \(\ell \colon A \to \Reals\).
  If there is a feasible potential \(y \colon V \to \Reals\) then
  \(D\) has no negative circuits.
\end{corollary}
\begin{proof}
  Let \(C\) be a circuit in~\(D\) from \(r\) to \(s = r\).  Then
  \(\ell(C) \geq y(s) - y(r) = 0\) by \Cref{thm:1}.
\end{proof}

\begin{corollary}
  \label{cor:1}
  Let \(D = (V,A)\) be a digraph on \(n\) vertices, let \(r \in V\),
  and let \(\ell \colon A \to \Reals\).  Suppose that, for each
  \(v \in V\), we have \(r \reaches_D v\).  If \(D\) has no negative
  circuit, then \(y_{n-1} = \dist_{(G,\ell)}(r,\cdot)\) is a feasible
  potential.
\end{corollary}
\begin{proof}
  This is a summary of what was discussed above.  Write the full proof
  as an exercise.
\end{proof}

Regarding the functions \(y_t\) defined in~\eqref{eq:7}
and~\eqref{eq:8}, it is easy to verify that \(y_{t} = y_{t+1}\) if and
only if \(y_t\) is a feasible potential.  Hence, to detect whether
\(D\) has a negative circuit, it suffices to check whether
\(y_{n-1} = y_n\) by \Cref{cor:1}.  To actually find a negative
circuit, suppose that \(y_n \neq y_{n-1}\).  Since
\(y_n \leq y_{n-1}\), there exists \(v \in V\) such that
\(y_n(v) < y_{n-1}(v)\).  By~\Cref{ex:5}, there is a walk~\(W\)
from~\(r\) to~\(v\) in~\(D\) of length exactly~\(n\) and cost
\(\ell(W) < \dist_{(G,\ell)}(r,v)\).  Hence, \(W\) must ``contain'' a
negative circuit.

\section{Aula de terça-feira, 21 de agosto}

\begin{theorem}[O algoritmo de Bellman-Ford]
  Sejam \(D = (V,A)\) um digrafo, \(r,s \in V\) e
  \(\ell \colon A \to \Reals\).  Defina \(y_t\) para cada
  \(t \in \Naturals\) como em~\eqref{eq:7} e~\eqref{eq:8} e defina
  \(W_t(v)\) para cada \(t \in \Naturals\) e \(v \in V\) como
  em~\eqref{eq:9} e~\eqref{eq:10}.  Defina
  \(R \coloneqq \setst{v \in V}{r \reaches_D v}\) e
  \(S \coloneqq \setst{v \in V}{v \reaches_D s}\).  Defina
  \(y^{\star} \coloneqq y_{n-1}\restrict{R \cap S}\), onde
  \(n \coloneqq \card{V}\).  Então vale precisamente uma das
  afirmações abaixo:
  \begin{enumerate}[(i)]
  \item O problema~\eqref{eq:3} é inviável e
    \(R = \setst{v \in V}{y_v^{\star} < \infty}\) satisfaz \(r \in R\),
    \(s \not\in R\) e \(\deltaout(R) = \emptyset\).
  \item O problema~\eqref{eq:3} é ilimitado e existe \(v \in S\) tal
    que o passeio \(W_n(v) = \seq{v_0,a_1,v_1,\dotsc,a_n,v_n}\) tem
    comprimento~\(n\) e, para quaisquer \(i,j \in \set{0,\dotsc,n}\)
    com \(i < j\) e \(v_i = v_j\), o passeio
    \(W_{ij} \coloneqq \seq{v_i,\dotsc,v_j}\) satisfaz
    \(\ell(W_{ij}) < 0\).
  \item O problema~\eqref{eq:3} tem valor ótimo
    \(y_s^{\star} \in \Reals\), o caminho
    \(P^{\star} \coloneqq W_{n-1}(s)\) é solução ótima
    para~\eqref{eq:3} e~\eqref{eq:2}, e \(y^{\star}\) é um potencial
    viável tal que \(\ell(P^{\star}) = y_s^{\star} - y_r^{\star}\).
  \end{enumerate}
\end{theorem}
\begin{proof}
  Novamente este é um resumo do que foi discutido acima.  Escreva a
  prova completa como exercício.
\end{proof}

\begin{exercise}[O algoritmo de Dijkstra]
  Sejam \(D = (V,A)\) um digrafo, \(r \in V\) e
  \(\ell \colon A \to \Reals_+\), onde
  \(\Reals_+ \coloneqq \setst{x \in \Reals}{x \geq 0}\).  Suponha que,
  para todo \(v \in V\), vale que \(r \reaches_D v\).  Defina \(y_0\)
  como em~\eqref{eq:2} acima e \(V_0 \coloneqq \emptyset\).  Para cada
  \(t \in \Naturals\), defina
  \begin{gather*}
    U_t
    \coloneqq
    \argmin\setst{y_t(v)}{v \in V \setminus V_t},
    \\
    V_{t+1}
    \coloneqq
    V_t \cup U_t,
    \\
    y_{t+1}(v)
    \coloneqq
    \begin{cases*}
      y_t(v)
      &
      se \(v \in V_{t+1}\),
      \\
      \min\paren[\big]{
        \set{y_t(v)}
        \cup
        \setst{y_t(u) + \ell(uv)}{uv \in A,\, u \in U_t}
      }
      &
      caso contrário,
    \end{cases*}
    \qquad
    \forall v \in V.
  \end{gather*}
  Prove que, para cada \(t \in \Naturals\), o potencial
  \(y_t\restrict{V_{t+1}}\) é viável no digrafo \(D[V_{t+1}]\).
\end{exercise}

Seja \(G = (V,E,\varphi)\) um grafo.  Para cada \(M \subseteq E\),
denote \(V_M \coloneqq \bigcup_{e \in E} \varphi(e)\).  Dizemos que
\(M \subseteq E\) é um \emph{emparelhamento} se
\(\card{V_M} = 2\card{M}\).  Note que \(M\) é um emparelhamento se, e
somente se, \(M\) não possui laços e, para quaisquer \(e,f \in M\)
distintos, vale que \(\varphi(e) \cap \varphi(f) = \emptyset\).  Neste
caso, os vértices de \(V_M\) são \emph{saturados} ou \emph{cobertos}
por~\(M\).  O emparelhamento \(M\) é \emph{perfeito} se \(V_M = V\).

\begin{problem*}[Problema do Emparelhamento de Peso Máximo]
  Dado um grafo \(G = (V,E,\varphi)\) e \(w \colon E \to \Reals\),
  resolva o problema de otimização
  \begin{equation}
    \label{eq:12}
    \begin{array}{{rl}}
      \text{Maximizar} & w(M) \\
      \text{sujeito a} & \text{\(M\) é um emparelhamento em \(G\)}.
    \end{array}
  \end{equation}
\end{problem*}

O valor ótimo de~\eqref{eq:12} é denotado por \(\nu_w(G)\).

\begin{problem*}[Problema do Emparelhamento Perfeito de Peso Mínimo]
  Dado um grafo \(G = (V,E,\varphi)\) e \(w \colon E \to \Reals\),
  resolva o problema de otimização
  \begin{equation}
    \label{eq:13}
    \begin{array}{{rl}}
      \text{Minimizar} & w(M) \\
      \text{sujeito a} & \text{\(M\) é um emparelhamento perfeito em \(G\)}.
    \end{array}
  \end{equation}
\end{problem*}

Vamos iniciar com o caso \(w = \ones\).  Nesse caso (de
cardinalidade), esses problemas se traduzem a: encontrar um
emparelhamento máximo e decidir se o grafo tem algum emparelhamento
perfeito.  Denote \(\nu \coloneqq \nu_{\ones}\).  Um emparelhamento
em~\(G\) de tamanho \(\nu(G)\) é \emph{máximo} (em~\(G\)).

Seja \(G = (V,E,\varphi)\) um grafo, \(M \subseteq E\) um
emparelhamento e \(P = \seq{v_0,e_1,v_1,\dotsc,e_k,v_k}\) um caminho
em~\(G\).  O caminho \(P\) é \emph{\(M\)-alternante} se, para cada
\(i \in [k-1]\), vale \(e_i \in M \iff e_{i+1} \not\in M\).  Se \(P\)
é \(M\)-alternante, então
\(V(P) \setminus \set{v_0,v_k} \subseteq V_M\); se \(v_0,v_k \not\in
V_M\), dizemos que \(P\) é \emph{\(M\)-aumentador}.

\begin{exercise}
  \label{ex:6}
  Seja \(G = (V,E,\varphi)\) um grafo, \(M \subseteq E\) um
  emparelhamento e \(P\) um caminho \(M\)-alternante de \(r\) a~\(s\)
  em~\(G\).  Prove que \(N \coloneqq M \mathbin{\Delta} E(P)\) é um
  emparelhamento e que \(V_N = V_M \mathbin{\Delta} \set{r,s}\).  Em
  particular, se \(P\) é \(M\)-aumentador, então \(M\) não é
  emparelhamento máximo.
\end{exercise}

\begin{theorem}[Teorema de Berge]
  Seja \(G = (V,E,\varphi)\) um grafo e \(M \subseteq E\) um
  emparelhamento.  Então \(M\) é emparelhamento máximo \(\iff\) não
  existe caminho \(M\)-aumentador em~\(G\).
\end{theorem}
\begin{proof}
  Necessidade segue imediatamente do \Cref{ex:6}.  Vamos provar a
  conversa da suficiência.  Suponha que \(M\) não é máximo, e seja
  \(N\) um emparelhamento tal que \(\card{N} > \card{M}\).  Se, para
  cada componente \(K\) de
  \(H \coloneqq (V, M \mathbin{\Delta} N, \varphi\restrict{M
    \mathbin{\Delta} N})\), temos
  \(\card{N \cap E[K]} \leq \card{M \cap E[K]}\), então
  \(\card{N} \leq \card{M}\).  Logo, existe um componente \(K\)
  de~\(H\) tal que \(N' \coloneqq N \cap E[K]\) tem mais arestas que
  \(M' \coloneqq M \cap E[K]\).  Seja
  \(P \coloneqq \seq{v_0,e_1,v_1,\dotsc,e_k,v_k}\) um caminho
  em~\(H[K]\) de comprimento máximo.  Não é difícil provar que
  (exercício!):
  \begin{enumerate}
  \item \(V(P) = K\) e \(E(P) = H[K]\);
  \item \(P\) é \(M\)- e \(N\)-alternante;
  \item \(v_1,\dotsc,v_{k-1} \in V_{M'} \cap V_{N'}\).
  \end{enumerate}
  Como
  \(\card{V_{N'}} > \card{V_{M'}}\) e ambos esses números são pares,
  segue que \(v_0,v_k \in V_{N'} \setminus V_{M'} \subseteq V_N
  \setminus V_M\) e portanto \(P\) é um caminho \(M\)-aumentador.
\end{proof}

Note que o teorema acima não fornece um certificado de otimalidade de
um emparelhamento.

Seja \(D = (V,A,\varphi)\) um digrafo.  O \emph{grafo subjacente}
a~\(D\) é o grafo \((V,A,\psi)\) onde
\begin{equation*}
  \psi(a)
  \coloneqq
  \begin{cases*}
    vv
    &
    se \(\varphi(a) = vv\),
    \\
    uv
    &
    se \(\varphi(a) = uv\).
  \end{cases*}
\end{equation*}
Seja \(G\) um grafo.  Um digrafo \(D\) é uma \emph{orientação}
de~\(G\) se \(G\) é o grafo subjacente a~\(D\).

Seja \(G = (V,E,\varphi)\) um grafo.  Dizemos que \(S\) é
\emph{estável} se \(G[S]\) não tem arestas.  Se \(U,W \subseteq V\)
são disjuntos e estáveis e vale que \(U \cup W = V\), dizemos que
\(G\) é \emph{\((U,W)\)-bipartido}.

Seja \(G = (V,E,\varphi)\) um grafo \((U,W)\)-bipartido.  Seja
\(M \subseteq E\) um emparelhamento e tome
\(\overline{M} \coloneqq E \setminus M\).  Defina o digrafo
\(D_M \coloneqq (V,E,\psi)\) onde
\begin{equation*}
  \psi(e)
  \coloneqq
  \begin{cases*}
    uw
    &
    se \(e \in \overline{M}\) e \(\varphi(e) = uw\) com \(u \in U\) e \(w \in W\),
    \\
    wu
    &
    se \(e \in M\) e \(\varphi(e) = uw\)  com \(u \in U\) e \(w \in W\),
  \end{cases*}
  \qquad
  \forall e \in E.
\end{equation*}

\begin{exercise}
  Seja \(G = (V,E,\varphi)\) um grafo \((U,W)\)-bipartido.  Seja
  \(M \subseteq E\) um emparelhamento.  Seja \(P\) um caminho em~\(G\)
  com início em \(u \in U\).  Prove que \(P\) é \(M\)-aumentador
  \(\iff\) \(P\) é um caminho em \(D_M\) de \(U \setminus V_M\) a
  \(V \setminus V_M\).
\end{exercise}

% Seja \(G = (V,E,\varphi)\) um grafo sem laços.  Defina
% \begin{equation*}
%   \delta_G(S)
%   \coloneqq
%   \setst{
%     e \in E
%   }{
%     \varphi(e) \cap S \neq \emptyset,\,
%     \varphi(e) \cap \overline{S} \neq \emptyset
%   }
%   \qquad
%   \forall S \subseteq V,
% \end{equation*}
% onde \(\overline{S} \coloneqq V \setminus S\) para todo
% \(S \subseteq V\).  O conjunto \(\delta_G(S)\) é o \emph{corte}
% de~\(G\) com \emph{margem}~\(S\).  Note que, para todo
% \(S \subseteq V\), vale \(\delta_G(S) = \delta_G(\overline{S})\).
% Abreviamos
% \begin{alignat*}{2}
%   \delta_G(v)
%   & \coloneqq
%   \delta_G(\set{v})
%   & \qquad &
%   \forall v \in V,
%   \\
%   \deg_G(v)
%   & \coloneqq
%   \card{\delta_G(v)}
%   & \qquad &
%   \forall v \in V.
% \end{alignat*}
% Denote \(\Delta(G) \coloneqq \max_{v \in V} \deg_G(v)\).

% Sejam \(G = (V,E)\) e \(H = (W,F)\) grafos.  Um \emph{homomorfismo}
% de~\(G\) para~\(H\) é uma função \(\pi \colon V \to W\) tal que, para
% todo \(e = \set{i,j} \in E\), vale que \(\set{\pi(i),\pi(j)} \in F\).
% Um \emph{isomorfismo} de~\(G\) para~\(H\) é uma função \(\pi\) que é
% um homomorfismo bijetor de~\(G\) para~\(H\) cuja inversa é um
% homomorfismo de~\(H\) para~\(G\).

\appendix

\section{Notação}

\bgroup
\renewcommand{\arraystretch}{1.2}
\begin{table}[!ht]
  \centering
  \begin{tabular}{r c p{12cm} }
    \toprule
    \([n]\)
    & \(\coloneqq\) &
    \(\set{1,\dotsc,n}\) para cada \(n \in \Naturals\)
    \\
    \(\Reals_+\)
    & \(\coloneqq\) &
    \(\setst{x \in \Reals}{x \geq 0}\), o conjunto dos reais não negativos
    \\
    \(\displaystyle\argmin_{x \in X} f(x)\)
    & \(\coloneqq\) &
    \(\setst{x^{\star} \in X}{\forall x \in X,\,f(x^*) \leq f(x)}\)
    \\
    \(A \mathbin{\Delta} B\)
    & \(\coloneqq\) &
    \((A \setminus B) \cup (B \setminus A)\),
    a \emph{diferença simétrica} entre conjuntos \(A\) e \(B\)
    \\
    \(\ones\)
    & \(\coloneqq\) &
    o vetor formado só por \(1\)'s no espaço apropriado
    \\
    \bottomrule
  \end{tabular}
  \label{tbl:special-sets}
\end{table}
\egroup                         % \arraystretch

\nocite{CookCPS98a,Schrijver03a}

\begingroup
\printbibliography
\endgroup

\end{document}

%%% Local Variables:
%%% mode: latex
%%% TeX-master: t
%%% coding: utf-8
%%% End:
