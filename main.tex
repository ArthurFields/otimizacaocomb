\documentclass[10pt,reqno]{amsart}

\usepackage{packages}

\title{MAC0325 Combinatorial Optimization: Lecture Notes}
\author{Instructor: Marcel {K.} de Carli Silva}
\date{August 13, 2018}
\address{%
  Instituto de Matemática e Estatística, %
  Universidade de São Paulo, %
  R. do Matão 1010, 05508-090, São Paulo, SP}
%
\email{mksilva@ime.usp.br}

\begin{document}

\begin{abstract}

\end{abstract}

\maketitle

\tableofcontents

\setcounter{section}{3}

\section*{Lecture 3, Tuesday, August 14}

To solve an optimization problem means to determine whether it is
\begin{enumerate}[(i)]
\item infeasible, or
\item unbounded, or
\item determine the optimal value and find an optimal solution, if one
  exists.
\end{enumerate}

\begin{problem}[The Shortest Paths Problem]
  Given a digraph \(D = (V,A,\varphi)\), vertices \(r,s \in V\), and
  \(\ell \colon A \to \Reals\), solve the optimization problem
  \begin{equation}
    \label{eq:1}
    \tag{SPP}
    \begin{array}{{rl}}
      \text{Minimize}   & \ell(P) \\
      \text{subject to} & \text{\(P\) is a path from \(r\) to \(s\) in~\(D\)}.
    \end{array}
  \end{equation}
\end{problem}

\begin{problem}[The Shortest Walks Problem]
  Given a digraph \(D = (V,A,\varphi)\), vertices \(r,s \in V\), and
  \(\ell \colon A \to \Reals\), solve the optimization problem
  \begin{equation}
    \label{eq:2}
    \tag{SWP}
    \begin{array}{{rl}}
      \text{Minimize}   & \ell(P) \\
      \text{subject to} & \text{\(P\) is a path from \(r\) to \(s\) in~\(D\)}.
    \end{array}
  \end{equation}
\end{problem}

It is obvious how to reduce any instance of either problem to one
where the digraph is simple.

\begin{proposition}
  Let \(D = (V,A,\varphi)\) be a digraph, let \(r,s \in V\), and let
  \(\ell \colon A \to \Reals\).  Then any optimal solution
  to~\eqref{eq:2} of minimum length is a path, and hence an optimal
  solution to~\eqref{eq:1}.
\end{proposition}

Let \(D = (V,A)\) be a digraph, let \(r,s \in V\), and let
\(\ell \colon A \to \Reals\).  For each \(t \in \Naturals\), define
\(y_t \colon V \to \Reals \cup \set{\infty}\) as
\begin{equation}
  \label{eq:3}
  y_0(v)
  \coloneqq
  \begin{cases*}
    0       & if \(v = r\); \\
    +\infty & otherwise,    \\
  \end{cases*}
\end{equation}
and
\begin{equation}
  \label{eq:4}
  y_{t+1}(v)
  \coloneqq
  \min\paren[\big]{
    \set{y_t(v)}
    \cup
    \setst{
      y_t(u) + \ell(uv)
    }{
      uv \in A
    }}
  \qquad
  \forall t \in \Naturals.
\end{equation}

\begin{exercise}
  Let \(D = (V,A)\) be a digraph, let \(r \in V\), and let
  \(\ell \colon A \to \Reals\).  Prove that, for each
  \(t \in \Naturals\) and each \(v \in V\), the smallest cost of a
  walk in~\(D\) from~\(r\) to~\(v\) with length \(\leq t\) is
  \(y_t(v)\).
\end{exercise}

It is easy to check that \(y_t\) can be computed in polynomial time
with respect to \(n = \card{V}\) and~\(t\).  A smart way to do this
computationally uses dynamic programming, and it is not hard to
recover the optimal walks from the recurrence relations
in~\eqref{eq:3} and~\eqref{eq:4}.

Let \(D = (V,A)\) be a digraph and let \(\ell \colon A \to \Reals\).
Let us temporarily call a function
\(y \colon V \to \Reals \cup \set{\infty}\) a \emph{potential}.  A
potential is \emph{feasible} if
\begin{equation*}
  y(v) \leq y(u) + \ell(uv)
  \qquad
  \forall uv \in A.
\end{equation*}
If \(y\) is a feasible potential and \(y(r) < \infty\), then
\(r \reaches_D v \implies y(v) < \infty\).  So there is very little
loss of generality when assume that a feasible potential is finite
everywhere.

\begin{theorem}
  \label{thm:1}
  Let \(D = (V,A)\) be a digraph, let \(r,s \in V\), and let
  \(\ell \colon A \to \Reals\).  If \(y \colon V \to \Reals\) is a
  feasible potential, then
  \begin{equation}
    \label{eq:5}
    \ell(W) \geq y(s) - y(r)
    \qquad
    \text{for any walk~\(W\) from~\(r\) to~\(s\) in~\(D\)}.
  \end{equation}
\end{theorem}
\begin{proof}
  Let \(W = \seq{v_0,a_1,v_1,\dotsc,a_k,v_k}\) be a walk from
  \(r = v_0\) to \(s = v_k\) in~\(D\).  Then
  \begin{equation*}
    \ell(W)
    =
    \sum_{i=1}^k \ell(a_i)
    =
    \sum_{i=1}^k \ell(v_{i-1}v_i)
    \geq
    \sum_{i=1}^k \sqbrac[\big]{y(v_i)-y(v_{i-1})}
    =
    y(v_k) - y(v_0)
    =
    y(s) - y(r)
  \end{equation*}
  since the sum telescopes.
\end{proof}

\begin{corollary}
  Let \(D = (V,A)\) be a digraph and let \(\ell \colon A \to \Reals\).
  If there is a feasible potential \(y \colon V \to \Reals\) then
  \(D\) has no negative circuits.
\end{corollary}
\begin{proof}
  Let \(C\) be a circuit in~\(D\) from \(r\) to \(s = r\).  Then
  \(\ell(C) \geq y(s) - y(r) = 0\) by \Cref{thm:1}.
\end{proof}

\nocite{CookCPS98a,Schrijver03a}

\begingroup
\printbibliography
\endgroup

\end{document}

%%% Local Variables:
%%% mode: latex
%%% TeX-master: t
%%% coding: utf-8
%%% End:
